\section[L and T]{\ifCDLeng Lemmas and Theorems\fi\ifCDLita Lemmi e Teoremi\fi}

\CDLeng We cite \emph{"Directed set --- {Wikipedia}{,} The Free Encyclopedia"} \cite{enwiki:1103032969}.
\CDLita Citiamo \emph{"Insieme diretto --- Wikipedia{,} L'enciclopedia libera"} \cite{itwiki:102088385}.

\begin{Lem}[Cauchy]\label{lem:1}
  \lipsum[3]

  See \cite{wiki:it:tautol}
\end{Lem}

\begin{Theorem}[\ifCDLeng The main theorem\fi\ifCDLita Il teorema principale\fi]\label{thm:1}
  \CDLeng Using Definition
  \CDLita Usando la definizione
  \ref{defn:1} \\
  \lipsum[4]
  \begin{proof}
    \CDLeng Using Lemma
    \CDLita Usando il Lemma
    \ref{lem:1}
    \begin{equation}
      \label{eq:1}
      {\mathbf{e}}^{i \pi }=-1
    \end{equation}
  \end{proof}
\end{Theorem}


\begin{Cor}\label{cor:1}
  \CDLeng Corollary to Theorem
  \CDLita Corollario al Teorema
  \ref{thm:1}.
  \\
  \lipsum[6]
\end{Cor}

\begin{description}
  % a nasty comment
\item[one] 1
\item[two] 2 %due
\item[three] 3
\end{description}


%%% Local Variables:
%%% mode: latex
%%% TeX-master: "paper"
%%% TeX-engine: xetex
%%% End:
